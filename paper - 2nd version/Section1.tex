\section {Introduction}
Shift-reduce parsing is one of the most classical method for syntactic parsing. In this approach, a parse tree is considered as a set of shift-reduce actions. In all possible parse trees whose scores are the sum of model scores assigned to its shift-reduce actions, the system would select the candidate having the highest score as a result. The approach of shift-reduce parsers is very successful. The advantage of this method is that it can exploit very rich features with a global decoding which leads to very high accuracies. \cite{2006Wang} present the first research on shift-reduce parser which can reach to 88\% F-score on English WSJ tree bank using SVM approach. In \cite{2009Zhang}, they has proposed out the global discriminative training for shift-reduce parsing which can achieve state-of-the-art performance for Chinese constituent parsing with the use of averaged perceptron from \cite{2004Collins}. Utilizing and extending the approach of \cite{2009Zhang}, \cite{2012Zhu} has built a fast and accurate constituent shift-reduce parser which even outperforms the previous state-of-the-art CYK-based parses such as Berkeley parser and Stanford parser.\\
\indent However, there is a problem which is still existed in those shift-reduce parsing system: all of them has to use inexact search to do the work in the decoding process. \cite{2006Wang} uses greedy search performing a deterministic decoder which must sacrifice a large space of candidate parse trees. \cite{2009Zhang} and \cite{2012Zhu} use beam search to extend the candidate space but it still cannot guarantee the exactness. We have a hypothesis that inexact search such as beam and greedy method may cause some searching errors and make the performance become worse than exact search. Therefore, in order to investigate the effect of this search errors, we would like to propose a strategy to perform an exact search for shift-reduce parsing by using A* heuristic. We apply our heuristic into the baseline shift-reduce constituent parsing of \cite{2009Zhang} to test our hypothesis. In our understanding, there is no previous research which can achieve the exactness on global discriminative training for shift-reduce parsing. Our final experiments on WSJ dataset has shown that our parser can overcome the previous constituent shift-reduce parsers in terms of F-score and even gives a comparable accuracy to the state-of-the-art parsers.
\subsection*{Related works}
\cite{2003DanNAACL} has proposed the first research on A* parsing for PCFG model. This system has maintained an agenda to store the processed nodes and extend them in a best-first order by using inside scores and an A* heuristic of outside scores. In their system, Klein and Manning summarized the context to calculate the A* heuristic of the outside probability. Therefore, they have to precomputed and stored the scores of all possible context summaries. They have reported that this heuristic can ignore more than 97\% of total possible nodes and lead the search reaching to goal very fast. However, A* heuristic is very difficult to be applied into the shift-reduce parsers. There are two main challenges in this approach: the first one is the very large search space caused by their rich features. The second one is that the model scores in shift-reduce parsing are always updated overtime due to the incremental training and cannot be directly precomputed as in \cite{2003DanNAACL}. \\
\indent Actually, there are still few parsers which try to perform exact search on shift-reduce parsing. \cite{2010Huang} uses dynamic programming to reduce the total time complexity to polynomial but they still have to use beam search in the decoding process. \cite{2006Sagae} and \cite{2013Zhao} has presented their ways of how to utilize best first search in shift-reduce parsing. However, these two system are locally trained on the maximum entropy model and their performances are still very far behind from the shift-reduce parsing system which has been trained by global discriminative method such as \cite{2012Zhu} or \cite{2009Zhang}. In our system, we firstly build the best first shift-reduce constituent parsing based on the dynamic algorithm of \cite{2010Huang}, and then apply our proposed A* heuristic to make the exact search in the shift-reduce approach become possible for the first time.