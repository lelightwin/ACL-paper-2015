\section{Basic Architecture}
	The goal of this section is to establish our basic parsing system other than our improvements explained in Section 4.
\subsection{Unary merging actions}
	\begin{itemize}
		\item Modifying the convention shift-reduce actions by merging UNARY action with SHIFT and B-REDUCE actions:
			\begin{itemize}
				\item SHIFT\_UNARY(X): in case of X=NULL, this is a conventional SHIFT action.
				\item B-REDUCE(Y)\_UNARY(X): in case of X=NULL, this is a conventional B-REDUCE action.
			\end{itemize}
		\item Advantage: 
			\begin{itemize}
				\item Normalize the number of actions for each parse tree will always be $2n$, with $n$ is the length of input sentence.
				\item It is similar to padding methods in \cite{2012Zhu}, but more consistent. In addition, the padding method in \cite{2012Zhu} cannot be applied into BFS.
				\item In dependency parsing, there is no need for using unary merging actions or padding method.
			\end{itemize}
	\end{itemize}
\subsection{Global Linear Best-First Parsing}
	\begin{itemize}
		\item The deductive system of our Best-First Parsing.
		\item Adding an offset to each action score to solve the negative-score problem.
	\end{itemize}
\subsection{Search Quality}
	\begin{itemize}
		\item Making comparison experiment between the search quality of Best-First Maxent and Best-First structured perceptron. The experiment has been taken on section 22 with 
		\item Based on the results, indicate that Zhao's parser had worked because Maxent model is sparser than Structured perceptron model.
	\end{itemize}